\documentclass{article}
\usepackage{url}
\usepackage[spacing,kerning]{microtype}
\usepackage[letterpaper]{geometry}
\usepackage{tocvsec2}
\settocdepth{subsection}

\usepackage{color}
\newcommand{\todo}[1]{\colorbox{red}{\begin{minipage}{\textwidth}{#1}\end{minipage}}}

\title{\todo{PhysioMIST Design Crap}}
\author{Mark Caral, Sara Cummins, BarbaraJoy Jones, Joshua Lee}
\date{October 23, 2009\\{\sc Eecs} 393}

\begin{document}

\begin{titlepage}
\maketitle\thispagestyle{empty}
\end{titlepage}

\todo{revision history}
Revision 0.1: October 2nd, 2009
\newline
\todo{Barbie: Intro, Use Case View}
\todo{Sara: make up stuff for the other sections}
\newpage

\tableofcontents
\newpage

\todo{make some totally bitchin' diagrams}
\todo{take out BS we don't actually BS}

\section{Introduction}
\subsection{Purpose}
\todo{give my life some meaning}
\subsection{Scope}
This document provides detailed description of the architecture of the PhysioMIST model integration interface.
\subsection{Glossary}
\todo{define crap}
JSim\\
MML\\
model\\
variable/parameter\\
FMA
\subsection{References}
\begin{enumerate}
\item \emph{Vision and Scope Document for PhysioMIST}
\item \emph{Software Requirements Specification for PhysioMIST}
\item \emph{PhysioMIST.} \url{http://robotics.case.edu/modeling_simulation_biological_systems.html}
\end{enumerate}
\subsection{Overview}
\todo{prettify me}
Architectural Representation: describes the views in the following sections\\
Architectural Goals and Constraints: describes the goals and constraints of the project's architecture\\
Use-Case View: describes typical use case scenarios and their impact on the architecture\\
Logical View: describes the one-to-one model integration use-case realizations; contains the Analysis and Design Models\\
Process View: describes concurrency aspects of the design\\
Deployment View: contains the Deployment Model\\
Implementation View: describes implementation details of the project's subsystems and functions\\
Data View: describes how the system handles persistent data; contains the Data Model\\
Size and Performance: describes performance issues and constraints\\
Quality: describes quality of service aspects of the project

\section{Architectural Representation}
\todo{overview crap}
\subsection{Logical View}
\textbf{Audience}: Designers\\
\textbf{Area}: Functional Requirements--describes the system's object model and most important use cases\\
\textbf{Related Items}: Design Model
\subsection{Implementation View}
\textbf{Audience}: Developers\\
\textbf{Area}: Software Components--describes implementation details of the project's subsystems and functions\\
\textbf{Related Items}: Implementation Model
\subsection{Use-Case View}
\textbf{Audience}: Project Team and end-users\\
\textbf{Area}: describes use cases and scenarios representing vital functionality\\
\textbf{Related Items}: Use-Case Model
\subsection{Data View}
\textbf{Audience}: Developers and end-users\\
\textbf{Area}: Persistence--describes how the system handles persistent data\\
\textbf{Related Items}: Data Model
\subsection{Process View}
\textbf{Audience}: Developers\\
\textbf{Area}: describes concurrency aspects of the design\\
\textbf{Related Items}: N/A
\subsection{Deployment View}
\todo{make some stuff up or delete this}
\textbf{Audience}:\\
\textbf{Area}:\\
\textbf{Related Items}:

\section{Architectural Goals and Constraints}
\todo{technical platform, data persistance, performance, i18n, other requirements}

\section{Use-Case View}
\todo{make up some use cases, BS an intro to the section}
\todo{this section needs some really BITCHIN' diagrams}
\subsection{Loading Models}
The user selects the ``Load'' menu item and then selects the file containing the desired model. The file may be a text file in the MML standard format or an XML file adhering to the PhysioMist schema. The software parses and validates the file then displays the model's data or gives an error if the file contents are not a valid model.
\subsection{Saving Models}
The user selects the ``Save'' or ``Save As...'' menu item. If the ``Save'' menu item is selected, the model is saved to the same file in the previously selected format. If the ``Save As...'' menu item is selected, the user must enter a file name and may optionally choose the file location and format. The default format is the PhysioMist XML format. If the user selects the ``Save'' menu item for an unsaved model, the ``Save As...'' dialog is presented.
\subsection{Editing Models}
\subsubsection{Adding Variables/Parameters}
The user clicks the ``New'' button for the variable or parameter table. A dialog box with fields for the name, formula (variables only), value, units, anatomical structure, and description is displayed. The user enters the relevant data, the input is validated, and the new item is added to the appropriate table. Only the name, formula, and value fields are required.
\todo{make sure this is correct}
\subsubsection{Deleting Variables/Parameters}
The user selects an item in the variable or parameter table and clicks the ``Delete'' button. The item is deleted from the model.
\subsubsection{Modifying Variables/Parameters}
The user selects an item in the variable or parameter table and clicks the ``Edit'' button. A dialog box as described above \todo{in the adding section} is displayed with the appropriate information in the relevant fields. The user modifies the data as needed, the input is validated, and the item is modified.
\subsubsection{Associating Anatomical Information}
\todo{check the GUI}
When the user is editing an item in the model, the user selects an anatomical structure in the Anatomical Ontology treeview and clicks the ``Associate Structure'' button. The structure is added to the item or replaces the previous structure.
\subsection{Model Integration}
This section is only concerned with integrating two models. The ability to integrate more than two models at a time may be a feature added in a future release.
\subsubsection{One-to-One Integration}
The user selects two models, which are loaded by the software as described above \todo{in the loading section}. The models will be referred to as $A$ and $B$ for clarity. The user then selects an item from each model ($A.x$ and $B.y$) and defines the relationship between them where $A.x$ acts as an input to $B.y$. This can be expressed mathematically as $F(A.x) = B.y$. Note that $A.x$ may be a variable or parameter, but $B.y$ must be a variable. The operations of the integration subsystem will not be addressed within this document.
\subsubsection{One-to-Many Integration}
This is very similar to one-to-one integration. If a relationship between $A.x$ and $B.y$ exists, the user is not restricted from defining a relationship with $A.x$ as input and $B.z$ as output. If $A.x$ is a variable and $B.w$ is not dependant upon $A.x$, the user may also define a relationship with $B.w$ as input to $A.x$. Circular relationships are handled by the underlying integration subsystem.
\subsubsection{Many-to-One Integration}
This feature may be added in a future release.
\subsection{Anatomical Queries}
\todo{list/explain the different relationship types}
\subsubsection{Structure-based Queries}
The user selects a structure in the Anatomical Ontology treeview and clicks the ``Related Structures'' menu item. The user then selects the type of relationship from the query dialog box. The software displays the results of the query (structures with the appropriate relationship to the selected structure) or an error if the relationship type is not applicable for the selected structure.
\subsubsection{Variable/Parameter-based Queries}
While integrating models, the user selects a variable/parameter from one model and clicks the ``Related Variables'' menu item. The user then selects the type of relationship from the query dialog box. The software displays the results of the query (variables/parameters from both models that are associated with structures that have the appropriate relationship to the selected item's associated anatomical structure) or an error if the relationship type is not applicable for the selected item's associated anatomical structure. An error is also displayed if the selected item does not have an associated anatomical structure.

\section{Logical View}
\todo{functional requirements, object model}

\section{Process View}
The software only executes one process at a time. The software is not multithreaded.

\section{Deployment View}
\todo{figure out what this should be. we aren't deploying anything. talk about users downloading the software?}

\section{Implementation View}
\todo{layers and subsystems}

\section{Data View}
\todo{describe saving/loading models, the XML schema, etc}

\section{Size and Performance}
\todo{"don't hang the GUI"}

\section{Quality}
\todo{pick buzzwords from Requirements powerpoint}

\end{document}