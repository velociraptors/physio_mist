\documentclass{article}
\usepackage{url}
\usepackage[spacing,kerning]{microtype}
\usepackage[letterpaper]{geometry}
\usepackage{tocvsec2}

\settocdepth{subsection}

\usepackage{color}
\newcommand{\todo}[1]{\colorbox{red}{\begin{minipage}{\textwidth}{#1}\end{minipage}}}

\title{User Manual\\
\bigskip
{\large for}\\
\bigskip
PhysioMIST}
\author{Mark Caral, Sara Cummins, BarbaraJoy Jones}
\date{December 04, 2009\\{\sc Eecs} 393}
\begin{document}

\begin{titlepage}
\maketitle\thispagestyle{empty}
\end{titlepage}

\tableofcontents
\newpage


\section{System Features}

\subsection{Import Existing Models}

To load an existing model: \\

\subsection{Save Anatomical Information}
\subsubsection{Description and Priority}
The system will save anatomical information associated with model components.\\
Priority = High.

\subsubsection{Stimulus/Response Sequences}
Stimulus: User requests to save anatomical information.\\
Response: Information is saved on user's hard drive.

\subsubsection{Functional Requirements}
Save.Request: Once the user has clicked the save button in the GUI, the system will save the anatomical information for the currently displayed model in a pre-determined location on the user's hard drive.

\subsection{Integrate Model Components on a One-to-One Basis}
\subsubsection{Description and Priority}
The system will provide a graphical interface for users to choose components from two models to integrate on a one-to-one basis.\\
Priority = High.

\subsubsection{Stimulus/Response Sequences}
Stimulus: User inputs two models to the system.\\
Response: System integrates the components of these models and displays the result.

\subsubsection{Functional Requirements}
OnetoOne.Integrate: The system will use the one-to-one algorithm to integrate the two models.

\subsection{Query for Related Model Components}
\subsubsection{Description and Priority}
The system will provide a graphical interface for users to search for related model components while integrating models, based on associated anatomical information.\\
Priority = High.
\subsubsection{Stimulus/Response Sequences}
Stimulus: User selects a model component with associated anatomical information and the type of query to perform.\\
Response: The system performs the query and displays the related model components.
\subsubsection{Functional Requirements}
Query.query: System selects related model components using the anatomical ontology.
Query.results: System displays the selected results.

\subsection{Integrate Model Components on a One-to-Many Basis}
\subsubsection{Description and Priority}
The system will provide a graphical interface for users to choose components from two models to integrate on a one-to-many basis.\\
Priority = Low.

\subsubsection{Stimulus/Response Sequences}
Stimulus: User inputs two models to the system and specifies the manner in which components should be integrated.\\
Response: System integrates the components of these models as specified and displays the result.

\subsubsection{Functional Requirements}
Onetomany.Integrate: The system will use the one-to-many algorithm to integrate the models.

\subsection{Integrate Model Components on a Many-to-One Basis}
\subsubsection{Description and Priority}
The system will provide a graphical interface for users to choose components from two models to integrate on a many-to-one basis.\\
Priority = Low.

\subsubsection{Stimulus/Response Sequences}
Stimulus: User inputs two models to the system and specifies the manner in which components should be integrated.\\
Response: System integrates the components of these models as specified and displays the result.

\subsubsection{Functional Requirements}
ManytoOne.Integrate: The system will use the many-to-one algorithm to integrate the models.

\subsection{Display Simulation Results}
\subsubsection{Description and Priority}
The system will provide a robust representation of the information contained in simulation results.\\
Priority = Low.

\subsubsection{Stimulus/Response Sequences}
Stimulus: User specifies the simulation parameters for a model.\\
Response: Display the results of the simulation graphically and textually.

\subsubsection{Functional Requirements}
Display.graph: The system will graphically show the results of a simulation.\\
Display.text: The system will display the simulation results as text.

\end{document}
