\documentclass{article}
\usepackage{url}
\usepackage[spacing,kerning]{microtype}
\usepackage[letterpaper]{geometry}

\usepackage{color}
\newcommand{\todo}[1]{\colorbox{red}{\begin{minipage}{\textwidth}{#1}\end{minipage}}}

\title{Software Requirements Specification\\
\bigskip
{\large for}\\
\bigskip
PhysioMIST}
\author{Mark Caral, Sara Cummins, BarbaraJoy Jones, Joshua Lee}
\date{September 23, 2009\\{\sc Eecs} 393}
\begin{document}

\begin{titlepage}
\maketitle\thispagestyle{empty}
\end{titlepage}

\tableofcontents
\newpage

\section{Introduction}
\subsection{Project Scope and Product Features}
\subsection{References}

\section{Overall Description}
\subsection{Product Perspective}
\subsection{User Classes and Characteristics}
\subsection{Operating Environment}
\subsection{Design and Implementation Constraints}
\subsection{User Documentation}
\subsection{Assumptions and Dependencies}

\section{System Features}

\subsection{Import Existing Models}

\subsubsection{Description and Priority}
The system will be able to import models that have been previously created in the JSim standard.\\
Priority = High.

\subsubsection{Stimulus/Response Sequences}
Stimulus: User locates a JSim file he or she wants to upload.\\
Response: System converts the file to its preferred format for further use.

\subsubsection{Functional Requirements}
Upload.check: Determine if the file is in the JSim format.\\
Upload.convert: Convert the file to the system's format.

\todo{etc. etc. Barbie, you already did this stuff, right?  Could you list whatever your functions were here so we don't have to change it later?}

\subsection{Save Anatomical Information}
\subsubsection{Description and Priority}
The system will save anatomical information associated with model components.\\
Priority = High.

\subsubsection{Stimulus/Response Sequences}
Stimulus: User requests to save anatomical information.\\
Response: Information is saved on user's hard drive.

\subsubsection{Functional Requirements}
Save.Request: Once the user has clicked the save button in the GUI, the system will save the anatomical information for the currently displayed model in a pre-determined location on the user's hard drive.

\subsection{Integrate Model Components on a One-to-One Basis}
\subsubsection{Description and Priority}
The system will provide a graphical interface for users to integrate components from two models on a one-to-one basis.\\
Priority = High.

\subsubsection{Stimulus/Response Sequences}
Stimulus: User inputs two models to the system.\\
Response: System integrates the components of these models and displays the result.

\subsubsection{Functional Requirements}
OnetoOne.Integrate: The system will use the one-to-one algorithm to integrate the two models.

\subsection{Integrate Model Components on a One-to-Many Basis}
\subsubsection{Description and Priority}
The system will provide a graphical interface for users to integrate components from several models on a one-to-many basis.\\
Priority = Low.

\subsubsection{Stimulus/Response Sequences}
Stimulus: User inputs multiple models to the system and specifies that the first should be integrated with all the rest.\\
Response: System integrates the components of these models and displays the result.

\subsubsection{Functional Requirements}
Onetomany.Integrate: The system will use the one-to-many algorithm to integrate the models.

\subsection{Integrate Model Components on a Many-to-One Basis}
\subsubsection{Description and Priority}
The system will provide a graphical interface for users to integrate components from several models on a many-to-one basis.\\
Priority = Low.

\subsubsection{Stimulus/Response Sequences}
Stimulus: User inputs multiple models to the system and specifies that the last should be integrated with all the rest.
Response: System integrates the components of these models and displays the result.

\subsubsection{Functional Requirements}
ManytoOne.Integrate: The system will use the many-to-one algorithm to integrate the models.

\subsection{Display Simulation Results}
\subsubsection{Description and Priority}
The system will provide a robust representation of the information contained in simulation results.\\
Priority = Low.

\subsubsection{Stimulus/Response Sequences}
Stimulus: User integrates models on a one-to-one, one-to-many, or many-to-one basis.\\
Response: Display the results of the comparison graphically.

\subsubsection{Functional Requirements}
Display.onetoone: The system will graphically show the results of a one-to-one comparison.\\
Display.onetomany: The system will graphically show the results of a one-to-many comparison.\\
Display.manytoone: The system will graphically show the results of a many-to-one comparison.

\section{External Interface Requirements}
\subsection{User Interface}
\subsection{Hardware Interface}
\subsection{Software Interface}
\subsection{Communications Interface}

\section{Other Nonfunctional Requirements}
\subsection{Performance Requirements}
\subsection{Safety Requirements}
\subsection{Security Requirements}
\subsection{Software Quality Requirements}


\end{document}
